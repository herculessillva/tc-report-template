\documentclass[conference]{IEEEtran}
\IEEEoverridecommandlockouts

% Pacotes utilizados
\usepackage[brazil]{babel}
\usepackage[utf8]{inputenc}
\usepackage{amsmath,amsfonts,amssymb}
\usepackage{graphicx}
\usepackage{float}
\usepackage{booktabs}
\usepackage{hyperref}
\usepackage{natbib}
\usepackage{geometry}
\geometry{left=3cm,right=2cm,top=2.5cm,bottom=2.5cm}

\begin{document}

% Título
\title{Título do Relatório da Disciplina}
\author{Nome do Aluno}
\date{\today}

\maketitle

\begin{abstract}
Este é um resumo breve do conteúdo do relatório. Deve conter a descrição sucinta dos objetivos, metodologia utilizada, resultados alcançados e as principais conclusões obtidas.
\end{abstract}

\section{Introdução}

Nesta seção, o aluno deve apresentar uma breve introdução ao tema abordado, objetivos do relatório e justificativa da relevância do estudo.

\section{Fundamentação Teórica}

Aqui devem ser apresentados os conceitos teóricos essenciais, com citações bibliográficas apropriadas. Como exemplo, segundo \citet{bishop2006pattern}, a fundamentação teórica é crucial para o desenvolvimento de pesquisas acadêmicas.

\section{Metodologia}

A metodologia utilizada deve ser descrita de forma detalhada. Pode-se incluir um diagrama ilustrativo (Figura \ref{fig:metodo}).

\begin{figure}[H]
    \centering
    \includegraphics[width=\linewidth]{figuras/fig1.png}
    \caption{Diagrama esquemático da metodologia utilizada.}
    \label{fig:metodo}
\end{figure}

\section{Resultados e Discussões}

Nesta seção, os principais resultados devem ser apresentados e discutidos. Exemplos incluem gráficos, tabelas ou descrições qualitativas dos resultados obtidos. Um exemplo de tabela está mostrado na Tabela \ref{tab:resultados}.

\begin{table}[H]
    \centering
    \caption{Exemplo de tabela com resultados obtidos.}
    \label{tab:resultados}
    \begin{tabular}{@{}lccc@{}}
        \toprule
        \textbf{Amostra} & \textbf{Média} & \textbf{Desvio Padrão} & \textbf{Observações} \\
        \midrule
        A & 20 & 2,5 & 30 \\
        B & 22 & 3,1 & 28 \\
        C & 19 & 1,8 & 32 \\
        \bottomrule
    \end{tabular}
\end{table}

\section{Conclusões}

Nesta seção, o aluno deve apresentar as principais conclusões obtidas a partir da análise dos resultados.

\bibliographystyle{apalike}
\bibliography{referencias}

\end{document}
